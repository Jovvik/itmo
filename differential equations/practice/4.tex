\input{preamble.sty}

\begin{document}

Есть теорема, которая гласит, что решение такого дифура имеет вид $y=\varphi + \gamma$, где $\gamma$ --- решение $y''-5y'=0$, а $\varphi$ --- частное решение искомого уравнения.

Найдём частное решение $y''-5y'=0$, пусть оно имеет вид $y = e^{Cx} \Rightarrow \begin{cases} y'= Ce^{Cx} \\ y'' = C^2e^{Cx} \end{cases}$

\begin{align*}
    y''-5y'            & = 0  \\
    C^2e^{Cx}-5Ce^{Cx} & = 0  \\
    C^2-5C             & = 0  \\
    \begin{cases}
        C = 0 \\
        C = 5
    \end{cases} \\
    \begin{cases}
        y = 1 \\
        y = e^{5x}
    \end{cases} \\
\end{align*}

Общее решение $y''-5y'=0$ имеет вид $y = C_1 y_1 + C_2 y_2 = C_1 + C_2e^{5x}$.

Теперь найдём частное решение искомого уравнения.

\begin{align*}
    y''-5y' & = 3x^2+\sin 5x \\
\end{align*}

Заметим, что слева все линейно, поэтому если $a$ --- решение $y'' - 5y' = 3x^2$,  а $b$ --- решение $y'' - y' = \sin 5x$, то их сумма --- (частное) решение искомого.

\begin{align*}
    a''-5a' & = 3x^2 \\
\end{align*}

Пусть $a$ --- полином, при этом его степень не больше 4:

\begin{align*}
    a   & = kx^4 + lx^3 + mx^2 + nx + p \\
    a'  & = 4kx^3 + 3lx^2 + 2mx + n     \\
    a'' & = 12kx^2 + 6lx + 2m           \\
\end{align*}

\begin{align*}
    12kx^2 + 6lx + 2m - 5(4kx^3 + 3lx^2 + 2mx + n) & = 3x^2 \\
\end{align*}

Очев $k=0$, т.е. полином степени 3.

\begin{align*}
    6lx + 2m - 5(3lx^2 + 2mx + n) & = 3x^2 \\
    \begin{cases}
        6l = 10m \\
        2m = 5n  \\
        -15l = 3 \\
    \end{cases}              \\
    \begin{cases}
        -0.12 = m  \\
        -0.048 = n \\
        l = -0.2   \\
    \end{cases}              \\
    a = -0.2x^3 - 0.12x^2 - 0.048x
\end{align*}

Найдём $b$:

\begin{align*}
    b''-5b' & = \sin 5x \\
\end{align*}

Предположим, что $b$ имеет вид $r\cos(5x) + s\sin(5x)$:

\begin{align*}
    b' = -5r\sin(5x) + 5s\cos(5x)    \\
    b'' = -25r\cos(5x) - 25s\sin(5x) \\
\end{align*}

\begin{align*}
    b''-5b'                                              & = \sin 5x \\
    25r\sin(5x) - 25s\cos(5x) -25r\cos(5x) - 25s\sin(5x) & = \sin 5x \\
    \begin{cases}
        25r - 25s = 1 \\
        -25s - 25r = 0
    \end{cases}                                       \\
    \begin{cases}
        50r = 1 \\
        s = -r
    \end{cases}                                       \\
    \begin{cases}
        r = 0.02 \\
        s = -0.02
    \end{cases}                                       \\
    b = -0.02\sin(5x) +0.02\cos(5x)
\end{align*}

Итого частное решение искомого дифура:

\[y = -0.2x^3 - 0.12x^2 - 0.048x -0.02\sin(5x) +0.02\cos(5x)\]

Ответ:
\[y = C_1 + C_2e^{5x} -0.2x^3 - 0.12x^2 - 0.048x -0.02\sin(5x) +0.02\cos(5x)\]

\end{document}