\documentclass[12pt, a4paper]{article}

% Math symbols
\usepackage{amsmath, amsthm, amsfonts, amssymb}
\usepackage{accents}
\usepackage{mathrsfs}
\usepackage{mathtools}

% Page layout
% \usepackage[left=1.5cm, right=1.5cm, top=2cm, bottom=2cm, bindingoffset=0cm, headheight=15pt]{geometry}
\usepackage{parskip, fancyhdr, a4wide}

% Font, encoding, russian support
\usepackage[russian]{babel}
\usepackage[sb]{libertine}
\usepackage{xltxtra}

% Miscellaneous
\usepackage{calc}
\usepackage{catchfilebetweentags}
\usepackage{enumitem}
\usepackage{etoolbox}
\usepackage{lastpage}
\usepackage{wrapfig}
\usepackage{xcolor}
\usepackage{bigints}

\providetoggle{useproofs}
\settoggle{useproofs}{false}

\pagestyle{fancy}
\lfoot{M3137y2019}
\rhead{стр. \thepage\ из \pageref{LastPage}}

\newcommand{\R}{\mathbb{R}}
\newcommand{\Q}{\mathbb{Q}}
\newcommand{\C}{\mathbb{C}}
\newcommand{\Z}{\mathbb{Z}}
\newcommand{\B}{\mathbb{B}}
\newcommand{\N}{\mathbb{N}}

\newcommand{\const}{\text{const}}

\newcommand{\teormin}{\textcolor{red}{!}\ }

\DeclareMathOperator*{\xor}{\oplus}
\DeclareMathOperator*{\equ}{\sim}
\DeclareMathOperator{\sign}{\text{sign}}
\DeclareMathOperator{\Sym}{\text{Sym}}
\DeclareMathOperator{\Asym}{\text{Asym}}

\DeclarePairedDelimiter{\ceil}{\lceil}{\rceil}
\DeclarePairedDelimiter{\abs}{\left\lvert}{\right\rvert}

\theoremstyle{plain}
\newtheorem{axiom}{Аксиома}
\newtheorem{lemma}{Лемма}[section]

\theoremstyle{remark}
\newtheorem*{remark}{Примечание}
\newtheorem*{exercise}{Упражнение}
\newtheorem*{corollary}{Следствие}
\newtheorem*{example}{Пример}
\newtheorem*{observation}{Наблюдение}

\theoremstyle{definition}
\newtheorem{theorem}{Теорема}[section]
\newtheorem*{definition}{Определение}
\newtheorem*{obozn}{Обозначение}

\newcommand{\dbltilde}[1]{\accentset{\approx}{#1}}
\newcommand{\intt}{\int\!}

% magical thing that fixes paragraphs
\makeatletter
\patchcmd{\CatchFBT@Fin@l}{\endlinechar\m@ne}{}
  {}{\typeout{Unsuccessful patch!}}
\makeatother

\newcommand{\get}[2]{
    \ExecuteMetaData[#1]{#2}
}

\newcommand{\getproof}[2]{
    \iftoggle{useproofs}{\ExecuteMetaData[#1]{#2proof}}{}
}

\newcommand{\getwithproof}[2]{
    \get{#1}{#2}
    \getproof{#1}{#2}
}

\newcommand{\import}[3]{
    \subsection{#1}
    \getwithproof{#2}{#3}
}

\newcommand{\given}[1]{
    Дано выше. (\ref{#1}, стр. \pageref{#1})
}

\renewcommand{\ker}{\text{Ker }}
\newcommand{\im}{\text{Im }}
\newcommand{\grad}{\text{grad}}
\newcommand{\defeq}{\stackrel{\text{def}}{=}}
\newcommand{\itemfix}{\leavevmode\makeatletter\makeatother}

\usepackage{sectsty}

\allsectionsfont{\raggedright}
\sectionfont{\normalfont\fontsize{14}{15}\selectfont}
\subsectionfont{\normalfont\fontsize{13}{15}\selectfont}
\usepackage{svg}
\svgpath{{images/}}
\allowdisplaybreaks

\makeatletter
\newenvironment{sqcases}{%
  \matrix@check\sqcases\env@sqcases
}{%
  \endarray\right.%
}
\def\env@sqcases{%
  \let\@ifnextchar\new@ifnextchar
  \left\lbrack
  \def\arraystretch{1.2}%
  \array{@{}l@{\quad}l@{}}%
}
\makeatother

\usepackage{tikz}
\usetikzlibrary{arrows.meta}
\usetikzlibrary{decorations.pathmorphing}
\usetikzlibrary{calc}
\usepackage{cases}

\begin{document}

Есть теорема, которая гласит, что решение такого дифура имеет вид $y=\varphi + \gamma$, где $\gamma$ --- решение $y''-5y'=0$, а $\varphi$ --- частное решение искомого уравнения.

Найдём частное решение $y''-5y'=0$, пусть оно имеет вид $y = e^{Cx} \Rightarrow \begin{cases} y'= Ce^{Cx} \\ y'' = C^2e^{Cx} \end{cases}$

\begin{align*}
    y''-5y'            & = 0  \\
    C^2e^{Cx}-5Ce^{Cx} & = 0  \\
    C^2-5C             & = 0  \\
    \begin{cases}
        C = 0 \\
        C = 5
    \end{cases} \\
    \begin{cases}
        y = 1 \\
        y = e^{5x}
    \end{cases} \\
\end{align*}

Общее решение $y''-5y'=0$ имеет вид $y = C_1 y_1 + C_2 y_2 = C_1 + C_2e^{5x}$.

Теперь найдём частное решение искомого уравнения.

\begin{align*}
    y''-5y' & = 3x^2+\sin 5x \\
\end{align*}

Заметим, что слева все линейно, поэтому если $a$ --- решение $y'' - 5y' = 3x^2$,  а $b$ --- решение $y'' - y' = \sin 5x$, то их сумма --- (частное) решение искомого.

\begin{align*}
    a''-5a' & = 3x^2 \\
\end{align*}

Пусть $a$ --- полином, при этом его степень не больше 4:

\begin{align*}
    a   & = kx^4 + lx^3 + mx^2 + nx + p \\
    a'  & = 4kx^3 + 3lx^2 + 2mx + n     \\
    a'' & = 12kx^2 + 6lx + 2m           \\
\end{align*}

\begin{align*}
    12kx^2 + 6lx + 2m - 5(4kx^3 + 3lx^2 + 2mx + n) & = 3x^2 \\
\end{align*}

Очев $k=0$, т.е. полином степени 3.

\begin{align*}
    6lx + 2m - 5(3lx^2 + 2mx + n) & = 3x^2 \\
    \begin{cases}
        6l = 10m \\
        2m = 5n  \\
        -15l = 3 \\
    \end{cases}              \\
    \begin{cases}
        -0.12 = m  \\
        -0.048 = n \\
        l = -0.2   \\
    \end{cases}              \\
    a = -0.2x^3 - 0.12x^2 - 0.048x
\end{align*}

Найдём $b$:

\begin{align*}
    b''-5b' & = \sin 5x \\
\end{align*}

Предположим, что $b$ имеет вид $r\cos(5x) + s\sin(5x)$:

\begin{align*}
    b' = -5r\sin(5x) + 5s\cos(5x)    \\
    b'' = -25r\cos(5x) - 25s\sin(5x) \\
\end{align*}

\begin{align*}
    b''-5b'                                              & = \sin 5x \\
    25r\sin(5x) - 25s\cos(5x) -25r\cos(5x) - 25s\sin(5x) & = \sin 5x \\
    \begin{cases}
        25r - 25s = 1 \\
        -25s - 25r = 0
    \end{cases}                                       \\
    \begin{cases}
        50r = 1 \\
        s = -r
    \end{cases}                                       \\
    \begin{cases}
        r = 0.02 \\
        s = -0.02
    \end{cases}                                       \\
    b = -0.02\sin(5x) +0.02\cos(5x)
\end{align*}

Итого частное решение искомого дифура:

\[y = -0.2x^3 - 0.12x^2 - 0.048x -0.02\sin(5x) +0.02\cos(5x)\]

Ответ:
\[y = C_1 + C_2e^{5x} -0.2x^3 - 0.12x^2 - 0.048x -0.02\sin(5x) +0.02\cos(5x)\]

\end{document}