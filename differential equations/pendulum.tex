\documentclass[12pt, a4paper]{article}

% Math symbols
\usepackage{amsmath, amsthm, amsfonts, amssymb}
\usepackage{accents}
\usepackage{cases}
\usepackage{mathrsfs}
\usepackage{mathtools}

% Page layout
% \usepackage[left=1.5cm, right=1.5cm, top=2cm, bottom=2cm, bindingoffset=0cm, headheight=15pt]{geometry}
\usepackage{parskip, fancyhdr, a4wide}

% Font, encoding, russian support
\usepackage[russian]{babel}
\usepackage[sb]{libertine}
\usepackage{xltxtra}

\usepackage{enumitem}
\usepackage{lipsum}
\usepackage{lastpage}
\usepackage{wrapfig}

\usepackage{svg}

\pagestyle{fancy}
\rhead{стр. \thepage\ из \pageref{LastPage}}

\lfoot{}
\cfoot{}
\rfoot{}

\title{Моделирование двойного маятника}
\author{Михайлов Максим, Кулагин Ярослав, Винников Глеб}

\begin{document}

\maketitle
\thispagestyle{fancy}

\begin{wrapfigure}{r}{0.5\textwidth}
    \centering
    \includesvg[width=0.48\textwidth]{images/Double-pendulum.svg}
    \caption{Двойной маятник}
\end{wrapfigure}

\textbf{Обозначения}:
\begin{itemize}[leftmargin=0pt]
    \item \(x_1,y_1,x_2,y_2\) --- координаты центров масс звеньев
    \item \(m_1, m_2\) --- массы звеньев
    \item \(l_1, l_2\) --- длины звеньев
    \item \(\theta_1, \theta_2\) --- угол отклонения от вертикали
\end{itemize}
% За точку отсчёта принимаем точку закрепления первого звена.

Выразим координаты центров:
\begin{numcases}{}
    x_1 = l_1 \sin \theta_1       \\
    y_1 = -l_1 \cos \theta_1      \\
    x_2 = x_1 + l_2 \sin \theta_2 \\
    y_2 = y_2 - l_2 \cos \theta_2
\end{numcases}

Возьмём первую и вторую производную этих выражений:

\begin{numcases}{}
    x_1' = l_1 \theta_1' \cos \theta_1        \\
    y_1' = l_1 \theta_1' \sin \theta_1        \\
    x_2' = x_1' + \theta_2' l_2 \cos \theta_2 \\
    y_2' = y_2' + \theta_2' l_2 \sin \theta_2
\end{numcases}

\begin{numcases}{}
    x_1'' = -l_1 \theta_1'^2 \sin \theta_1 + l_1 \theta_1'' \cos \theta_1 \label{ускорение x1}        \\
    y_1'' = l_1 \theta_1'^2 \cos \theta_1 + l_1 \theta_1'' \sin \theta_1 \label{ускорение y1}         \\
    x_2'' = x_1'' - \theta_2'^2 l_2 \sin \theta_2 + \theta_2'' l_2 \cos \theta_2 \label{ускорение x2} \\
    y_2'' = y_1'' + \theta_2'^2 l_2 \cos \theta_2 + \theta_2'' l_2 \sin \theta_2 \label{ускорение y2}
\end{numcases}

\begin{wrapfigure}{r}{0.3\textwidth}
    \centering
    \includesvg[width=0.28\textwidth]{images/Double-pendulum-force1.svg}
    \caption{Силы в двойном маятнике}
\end{wrapfigure}

По второму закону Ньютона \textit{(в проекциях)} для верхнего звена:
\begin{numcases}{}
    m_1x_1'' = - T_1 \sin \theta_1 + T_2 \sin \theta_2 \label{ньютон верх1} \\
    m_1y_1'' = T_1 \cos \theta_1 - T_2 \cos \theta_2 - m_1g \label{ньютон верх2}
\end{numcases}

Аналогично для нижнего звена:

\begin{numcases}{}
    m_2x_2'' = - T_2 \sin \theta_2 \label{ньютон низ1} \\
    m_2y_2'' = T_2 \cos \theta_2 - m_2g \label{ньютон низ2}
\end{numcases}

Выразим \(T_2\sin \theta_2\) и \(T_2\cos \theta_2\) из \eqref{ньютон низ1}, \eqref{ньютон низ2} и подставим в \eqref{ньютон верх1}, \eqref{ньютон верх2}

\begin{numcases}{}
    m_1 x_1'' = - T_1 \sin \theta_1 - m_2x_2'' \label{a} \\
    m_1 y_1'' = T_1\cos\theta_1 - m_2y_2'' - m_2g - m_1g \label{b}
\end{numcases}

Домножим \eqref{a} на \(\cos \theta_1\), \eqref{b} на \(\sin \theta_1\):

\begin{equation}
    \begin{cases}
        m_1 x_1''\cos \theta_1 = - T_1 \sin \theta_1\cos \theta_1 - m_2x_2''\cos \theta_1 \\
        m_1 y_1''\sin \theta_1 = T_1\cos\theta_1\sin \theta_1 - \sin \theta_1(m_2y_2'' + m_2g + m_1g)
    \end{cases}
\end{equation}
\begin{equation}
    \begin{cases}
        T_1\sin \theta_1 \cos \theta_1 =- \cos \theta_1 (m_1x_1'' + m_2x_2'') \\
        T_1\sin \theta_1\cos \theta_1 = \sin\theta_1 (m_1y_1'' + m_2y_2'' + m_2g + m_1g)
    \end{cases}
\end{equation}

\begin{equation}
    \sin\theta_1 (m_1y_1'' + m_2y_2'' + m_2g + m_1g) = - \cos \theta_1 (m_1x_1'' + m_2x_2'') \label{c}
\end{equation}

Домножим \eqref{ньютон низ1} на \(\cos \theta_2\) и \eqref{ньютон низ2} на \(\sin \theta_2\):

\begin{equation}
    \begin{cases}
        m_2x_2''\cos \theta_2 = - T_2 \sin \theta_2\cos \theta_2 \\
        m_2y_2''\sin \theta_2 = T_2 \sin \theta_2\cos \theta_2 - m_2g\sin \theta_2
    \end{cases} \label{d}
\end{equation}

Приравняем правые части \eqref{d}:
\begin{equation}
    \sin \theta_2 (m_2y_2'' + m_2g) = - \cos \theta_2(m_2x_2'')
\end{equation}
\begin{equation}
    \sin \theta_2 (y_2'' + g) = - \cos \theta_2(x_2'')
\end{equation}
Подставим \(y_2''\) и \(x_2''\) из \eqref{ускорение x2} и \eqref{ускорение y2}:
\begin{equation}
    \sin \theta_2 (y_1'' + \theta_2'^2 l_2 \cos \theta_2 + \theta_2'' l_2 \sin \theta_2 + g) = - \cos \theta_2(x_1'' - \theta_2'^2 l_2 \sin \theta_2 + \theta_2'' l_2 \cos \theta_2)
\end{equation}
Подставим \(y_1''\) и \(x_1''\) из \eqref{ускорение x1} и \eqref{ускорение y1}:
\begin{align}
    \sin \theta_2 (l_1 \theta_1'^2 \cos \theta_1 + l_1 \theta_1'' \sin \theta_1 + \theta_2'^2 l_2 \cos \theta_2 + \theta_2'' l_2 \sin \theta_2 + g) & = \label{f} \\
    - \cos \theta_2(-l_1 \theta_1'^2 \sin \theta_1 + l_1 \theta_1'' \cos \theta_1 - \theta_2'^2 l_2 \sin \theta_2 + \theta_2'' l_2 \cos \theta_2) \nonumber
\end{align}
\begin{equation}
    \theta_1'^2 l_1 (\sin \theta_2 \cos \theta_1 - \sin \theta_1 \cos \theta_2) + \theta_1'' l_1 (\sin \theta_1 \sin \theta_2 + \cos \theta_1 \cos \theta_2) + \theta_2'' l_2 + \sin \theta_2 g = 0
\end{equation}
\begin{equation}
    \theta_2'' = -\frac{l_1}{l_2} (\theta_1'^2 (\sin \theta_2 \cos \theta_1 - \sin \theta_1 \cos \theta_2) + \theta_1'' (\sin \theta_1 \sin \theta_2 + \cos \theta_1 \cos \theta_2)) - \sin \theta_2 \frac{g}{l}
\end{equation}
\begin{equation}
    \theta_2'' = \frac{l_1}{l_2} (\theta_1'^2 \sin (\theta_1 - \theta_2) - \theta_1'' \cos(\theta_2 - \theta_1)) - \sin \theta_2 \frac{g}{l_2} \label{тета2}
\end{equation}

Подставим в \eqref{c} \(y_2''\) и \(x_2''\) из \eqref{ускорение x2} и \eqref{ускорение y2}
\begin{align}
    \sin\theta_1 (m_1y_1'' + m_2(y_1'' + \theta_2'^2 l_2 \cos \theta_2 + \theta_2'' l_2 \sin \theta_2) + m_2g + m_1g) & = \\
    - \cos \theta_1 (m_1x_1'' + m_2(x_1'' - \theta_2'^2 l_2 \sin \theta_2 + \theta_2'' l_2 \cos \theta_2)) \nonumber
\end{align}

Подставим \(y_1''\) и \(x_1''\) из \eqref{ускорение x1} и \eqref{ускорение y1}:
\begin{align*}
    \sin\theta_1 ((m_1 + m_2)(l_1 \theta_1'^2 \cos \theta_1 + l_1 \theta_1'' \sin \theta_1) + m_2(\theta_2'^2 l_2 \cos \theta_2 + \theta_2'' l_2 \sin \theta_2) + m_2g + m_1g) & = \\
    - \cos \theta_1 ((m_1 + m_2)(-l_1 \theta_1'^2 \sin \theta_1 + l_1 \theta_1'' \cos \theta_1) + m_2(- \theta_2'^2 l_2 \sin \theta_2 + \theta_2'' l_2 \cos \theta_2))
\end{align*}
\begin{align*}
    \theta_1''l_1(m_1 + m_2) + \theta_1'' \cdot 0 + \theta_2'' l_2 m_2(\sin \theta_1 \sin \theta_2 + \cos \theta_1 \cos \theta_2) & {}+{} \\
    \theta_2'^2 l_2m_2 (\sin \theta_1 \cos \theta_2 - \sin \theta_2 \cos \theta_1) + \sin \theta_1 g(m_1 + m_2)                   & = 0
\end{align*}
\begin{equation*}
    \theta_1''l_1(m_1 + m_2) + \theta_2'' l_2 m_2 \cos(\theta_2 - \theta_1) - \theta_2'^2 l_2m_2 \cos(\theta_1 + \theta_2) + \sin \theta_1 g(m_1 + m_2) = 0
\end{equation*}
\begin{equation}
    \theta_1'' = \theta_2'^2 \frac{l_2m_2}{l_1(m_1 + m_2)}  \cos(\theta_1 + \theta_2) - \theta_2'' \frac{l_2 m_2}{l_1(m_1 + m_2)} \cos(\theta_2 - \theta_1) - \sin \theta_1 \frac{g}{l_1} \label{тета1}
\end{equation}

Подставим \eqref{тета1} в \eqref{тета2}:

\begin{align*}
    \theta_2'' = \frac{l_1}{l_2} (\theta_1'^2 \sin (\theta_1 - \theta_2) - \left( \theta_2'^2 \frac{l_2m_2}{l_1(m_1 + m_2)}  \cos(\theta_1 + \theta_2) - \theta_2'' \frac{l_2 m_2}{l_1(m_1 + m_2)} \cos(\theta_2 - \theta_1) - \sin \theta_1 \frac{g}{l_1} \right) & {}\cdot{} \\
    \cos(\theta_2 - \theta_1)) - \sin \theta_2 \frac{g}{l_2}
\end{align*}
\begin{align*}
    \theta_2'' = \frac{l_1}{l_2}\theta_1'^2 \sin (\theta_1 - \theta_2) - \left(\theta_2'^2 \frac{m_2}{m_1 + m_2}  \cos(\theta_1 + \theta_2) - \theta_2'' \frac{m_2}{m_1 + m_2} \cos(\theta_2 - \theta_1) - \sin \theta_1 \frac{g}{l_2} \right) & {}\cdot{} \\
    \cos(\theta_2 - \theta_1)) - \sin \theta_2 \frac{g}{l_2}
\end{align*}
\begin{align*}
    \theta_2'' = \frac{l_1}{l_2}\theta_1'^2 \sin (\theta_1 - \theta_2) - \theta_2'^2 \frac{m_2}{m_1 + m_2} \frac{1}{2} (\cos(2\theta_1) + \cos(2\theta_2)) & {}-{} \\
    \theta_2'' \frac{m_2}{m_1 + m_2} \cos^2(\theta_2 - \theta_1) - \cos(\theta_2 - \theta_1)\sin \theta_1 \frac{g}{l_2} - \sin \theta_2 \frac{g}{l_2}
\end{align*}
\begin{align*}
    \theta_2''\frac{2m_1 + 2m_2 - m_2 \cos^2(\theta_2 - \theta_1)}{m_1 + m_2} = \frac{l_1}{l_2}\theta_1'^2 \sin (\theta_1 - \theta_2) - \theta_2'^2 \frac{m_2}{m_1 + m_2} \frac{1}{2} (\cos(2\theta_1) + \cos(2\theta_2)) {}-{} \\
    \cos(\theta_2 - \theta_1)\sin \theta_1 \frac{g}{l_2} - \sin \theta_2 \frac{g}{l}
\end{align*}
\begin{align*}
    \theta_2''\frac{2m_1 + m_2 - m_2 \cos(2\theta_2 - 2\theta_1)}{2m_1 + 2m_2} = \frac{l_1}{l_2}\theta_1'^2 \sin (\theta_1 - \theta_2) - \theta_2'^2 \frac{m_2}{m_1 + m_2} \frac{1}{2} (\cos(2\theta_1) + \cos(2\theta_2)) {}-{} \\
    \cos(\theta_2 - \theta_1)\sin \theta_1 \frac{g}{l_2} - \sin \theta_2 \frac{g}{l}
\end{align*}
\begin{align*}
    \theta_2''\frac{2m_1 + m_2 - m_2\cos(2\theta_2 - 2\theta_1)}{2m_1 + 2m_2} & = \frac{l_1^2(m_1 + m_2)\theta_1'^2\sin (\theta_1 -\theta_2) - m_2\theta_2'^2 l_1l_2\sin(\theta_2 -\theta_1)\cos(\theta_2 -\theta_1)}{l_2l_1(m_1 + m_2)} \\
                                                                              & + \frac{l_1\sin \theta_1 \cos(\theta_1 -\theta_2)g(m_1 + m_2) - l_1(m_1 + m_2)\sin \theta_2 g}{l_2l_1(m_1 + m_2)}
\end{align*}
\begin{align*}
    \theta_2''\frac{2m_1 + m_2 - m_2\cos(2\theta_2 - 2\theta_1)}{2} & = \frac{l_1(m_1 + m_2)\theta_1'^2\sin (\theta_1 -\theta_2) - m_2\theta_2'^2 l_2\sin(\theta_2 -\theta_1)\cos(\theta_2 -\theta_1)}{l_2} \\
                                                                    & + \frac{\sin \theta_1 \cos(\theta_1 -\theta_2)g(m_1 + m_2) - (m_1 + m_2)\sin \theta_2 g}{l_2}
\end{align*}
\begin{align*}
    \theta_2'' & = 2\frac{l_1(m_1 + m_2)\theta_1'^2\sin (\theta_1 -\theta_2) - m_2\theta_2'^2 l_2\sin(\theta_2 -\theta_1)\cos(\theta_2 -\theta_1)}{l_2(2m_1 + m_2 - m_2\cos(2\theta_2 - 2\theta_1))} \\
               & + 2\frac{\sin \theta_1 \cos(\theta_1 -\theta_2)g(m_1 + m_2) - (m_1 + m_2)\sin \theta_2 g}{l_2(2m_1 + m_2 - m_2\cos(2\theta_2 - 2\theta_1))}
\end{align*}
\begin{align}
    \theta_2'' & = 2\sin(\theta_1 - \theta_2)\frac{l_1(m_1 + m_2)\theta_1'^2 + m_2\theta_2'^2 l_2\cos(\theta_1 -\theta_2) + g(m_1 + m_2)\cos \theta_1}{l_2(2m_1 + m_2 - m_2\cos(2\theta_2 - 2\theta_1))} \label{fin2}
\end{align}
\begin{align*}
    \theta_1'' & = \frac{m_2 \theta_2'^2 l_2\sin (\theta_2 - \theta_1) - m_2(l_1 (\theta_1'^2 \sin (\theta_1 - \theta_2) - \theta_1'' \cos (\theta_1 - \theta_2)) - \sin \theta_2 g)\cos (\theta_2 - \theta_1)}{l_1(m_1 + m_2)} \\
               & + \sin \theta_1 \frac{g}{l_1}
\end{align*}
\begin{align*}
    \theta_1'' - \frac{m_2 \theta_1'' \cos (\theta_1 - \theta_2)}{m_1 + m_2} & = \frac{m_2 \theta_2'^2 l_2\sin (\theta_2 - \theta_1) - m_2(l_1 \theta_1'^2 \sin (\theta_1 - \theta_2) - \sin \theta_2 g)\cos (\theta_2 - \theta_1)}{l_1(m_1 + m_2)} \\
                                                                             & + \sin \theta_1 \frac{g}{l_1}
\end{align*}
\begin{align}
    \theta_1'' & = \frac{2m_2 \sin (\theta_2 - \theta_1)(\theta_2' l_2 + \theta_1'^2 l_1 \cos(\theta_1 - \theta_2)) + m_2\sin (2\theta_2 - \theta_1)g + g \sin \theta_1 (m_2 - 2m_1 - 2m_2)}{l_1(2m_1 + m_2 - m_2\cos (2\theta_2 - 2\theta_1))} \nonumber \\
               & = \frac{2m_2 \sin (\theta_2 - \theta_1)(\theta_2' l_2 + \theta_1'^2 l_1 \cos(\theta_1 - \theta_2)) + m_2\sin (2\theta_2 - \theta_1)g - g \sin \theta_1 (2m_1 + m_2)}{l_1(2m_1 + m_2 - m_2\cos (2\theta_2 - 2\theta_1))} \label{fin1}
\end{align}

Дифуры \eqref{fin1} и \eqref{fin2} решаются методом Рунге-Кутты, если ввести функции \(w_1(t) = \theta_1'(t)\) и \(w_2(t) = \theta_2'(t)\) \textit{(угловые скорости)}.

\end{document}