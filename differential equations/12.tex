\documentclass[12pt, a4paper]{article}

% Math symbols
\usepackage{amsmath, amsthm, amsfonts, amssymb}
\usepackage{accents}
\usepackage{mathrsfs}
\usepackage{mathtools}

% Page layout
% \usepackage[left=1.5cm, right=1.5cm, top=2cm, bottom=2cm, bindingoffset=0cm, headheight=15pt]{geometry}
\usepackage{parskip, fancyhdr, a4wide}

% Font, encoding, russian support
\usepackage[russian]{babel}
\usepackage[sb]{libertine}
\usepackage{xltxtra}

% Miscellaneous
\usepackage{calc}
\usepackage{catchfilebetweentags}
\usepackage{enumitem}
\usepackage{etoolbox}
\usepackage{lastpage}
\usepackage{wrapfig}
\usepackage{xcolor}
\usepackage{bigints}

\providetoggle{useproofs}
\settoggle{useproofs}{false}

\pagestyle{fancy}
\lfoot{M3137y2019}
\rhead{стр. \thepage\ из \pageref{LastPage}}

\newcommand{\R}{\mathbb{R}}
\newcommand{\Q}{\mathbb{Q}}
\newcommand{\C}{\mathbb{C}}
\newcommand{\Z}{\mathbb{Z}}
\newcommand{\B}{\mathbb{B}}
\newcommand{\N}{\mathbb{N}}

\newcommand{\const}{\text{const}}

\newcommand{\teormin}{\textcolor{red}{!}\ }

\DeclareMathOperator*{\xor}{\oplus}
\DeclareMathOperator*{\equ}{\sim}
\DeclareMathOperator{\sign}{\text{sign}}
\DeclareMathOperator{\Sym}{\text{Sym}}
\DeclareMathOperator{\Asym}{\text{Asym}}

\DeclarePairedDelimiter{\ceil}{\lceil}{\rceil}
\DeclarePairedDelimiter{\abs}{\left\lvert}{\right\rvert}

\theoremstyle{plain}
\newtheorem{axiom}{Аксиома}
\newtheorem{lemma}{Лемма}[section]

\theoremstyle{remark}
\newtheorem*{remark}{Примечание}
\newtheorem*{exercise}{Упражнение}
\newtheorem*{corollary}{Следствие}
\newtheorem*{example}{Пример}
\newtheorem*{observation}{Наблюдение}

\theoremstyle{definition}
\newtheorem{theorem}{Теорема}[section]
\newtheorem*{definition}{Определение}
\newtheorem*{obozn}{Обозначение}

\newcommand{\dbltilde}[1]{\accentset{\approx}{#1}}
\newcommand{\intt}{\int\!}

% magical thing that fixes paragraphs
\makeatletter
\patchcmd{\CatchFBT@Fin@l}{\endlinechar\m@ne}{}
  {}{\typeout{Unsuccessful patch!}}
\makeatother

\newcommand{\get}[2]{
    \ExecuteMetaData[#1]{#2}
}

\newcommand{\getproof}[2]{
    \iftoggle{useproofs}{\ExecuteMetaData[#1]{#2proof}}{}
}

\newcommand{\getwithproof}[2]{
    \get{#1}{#2}
    \getproof{#1}{#2}
}

\newcommand{\import}[3]{
    \subsection{#1}
    \getwithproof{#2}{#3}
}

\newcommand{\given}[1]{
    Дано выше. (\ref{#1}, стр. \pageref{#1})
}

\renewcommand{\ker}{\text{Ker }}
\newcommand{\im}{\text{Im }}
\newcommand{\grad}{\text{grad}}
\newcommand{\defeq}{\stackrel{\text{def}}{=}}
\newcommand{\itemfix}{\leavevmode\makeatletter\makeatother}

\usepackage{sectsty}

\allsectionsfont{\raggedright}
\sectionfont{\normalfont\fontsize{14}{15}\selectfont}
\subsectionfont{\normalfont\fontsize{13}{15}\selectfont}
\usepackage{svg}
\svgpath{{images/}}
\allowdisplaybreaks

\makeatletter
\newenvironment{sqcases}{%
  \matrix@check\sqcases\env@sqcases
}{%
  \endarray\right.%
}
\def\env@sqcases{%
  \let\@ifnextchar\new@ifnextchar
  \left\lbrack
  \def\arraystretch{1.2}%
  \array{@{}l@{\quad}l@{}}%
}
\makeatother

\usepackage{tikz}
\usetikzlibrary{arrows.meta}
\usetikzlibrary{decorations.pathmorphing}
\usetikzlibrary{calc}
\usepackage{cases}

\lhead{Домашнее задание №12}
\lfoot{Михайлов Максим}
\cfoot{}
\rfoot{M3237}

\begin{document}

\section{}

Линейной заменой уничтожить член с первой производной. \(x^2y''-2xy'+(x^2+2)y=0\). Указать через пробел замену и получившиеся уравнение.

\[
    y : = a(x) z \quad y' = a'z + az' \quad y'' = a''z + az'' + 2a'z'
\]

\begin{align*}
    x^2y''-2xy'+(x^2+2)y\,                                     & = 0            \\
    x^2(a''z + az'' + 2a'z') - 2x(a'z + az') + (x^2+2) az      & = 0            \\
    x^2a''z + x^2az'' + 2x^2a'z' - 2xa'z - 2xaz' + x^2az + 2az & = 0            \\
    x^2a''z + x^2az'' + z'(2x^2a' - 2xa) - 2xa'z + x^2az + 2az & = 0            \\
    2x^2 a' - 2xa                                              & = 0            \\
    x a' - a                                                   & = 0            \\
    a'                                                         & = \frac{a}{x}  \\
    \frac{da}{a}                                               & = \frac{dx}{x} \\
    \ln|a|                                                     & = \ln|x| + C   \\
    |a|                                                        & = |x|e^C       \\
    a                                                          & = \pm xe^C     \\
    a                                                          & = xC           \\
    a'                                                         & = C            \\
    a''                                                        & = 0            \\
    x^2z''xC - 2xzC + x^2zxC + 2zxC                            & = 0            \\
    x^2z''xC + x^2zxC                                          & = 0            \\
    x^3C(z'' + z)                                              & = 0            \\
\end{align*}

\section{}

Заменой независимого переменного уничтожить член с первой производной. \(xy''-y'-4x^3y=0\). Указать через пробел замену и получившиеся уравнение.

\[
    t : = \varphi(x) \quad y'_x = \varphi'_x y'_t \quad y''_{xx} = \varphi''_{xx} y'_t + (\varphi'_x)^2 y''_{tt}
\]

\begin{align*}
    xy''-y'-4x^3y & = 0 \\
\end{align*}
Далее все производные \(y\) берутся по \(t\), все производные \(\varphi\) --- по \(x\)
\begin{align*}
    x(\varphi'' y' + \varphi'^2 y'')-\varphi' y' - 4x^3y & = 0                \\
    x\varphi'' y' + x\varphi'^2 y''-\varphi' y' - 4x^3y  & = 0                \\
    x\varphi'^2 y'' + (x\varphi'' - \varphi') y' - 4x^3y & = 0                \\
    x\varphi'' - \varphi'                                & = 0                \\
    x\varphi''                                           & = \varphi'         \\
    t : = \varphi'                                                            \\
    xt'                                                  & = t                \\
    t                                                    & = xC               \\
    \varphi'                                             & = xC               \\
    \varphi                                              & = \frac{xC}{2} + A \\
    \varphi''                                            & = C                \\
    x^3C^2 y'' - 4x^3y                                   & = 0                \\
\end{align*}

\section{}

Найти линейно независимое решение в виде ряда. \((1 - x^2)y'' - 4xy' - 2y = 0\). Собрать ряды в функцию. Если ответов несколько указать через пробел.

Пусть \(y = \sum\limits_{n = 0}^{ +\infty} a_n x^n\)

\begin{align*}
    (1 - x^2)y'' - 4xy' - 2y                                                                                                                                                  & = 0 \\
    (1 - x^2)\sum\limits_{n = 2}^{ +\infty} n(n - 1) a_n x^{n - 2} - 4x\sum\limits_{n = 1}^{ +\infty} na_n x^{n - 1} - 2\sum\limits_{n = 0}^{ +\infty} a_n x^n                & = 0 \\
    \sum\limits_{n = 2}^{ +\infty} n(n - 1) a_n (x^{n - 2} - x^n) - 4\sum\limits_{n = 1}^{ +\infty} na_n x^n - 2\sum\limits_{n = 0}^{ +\infty} a_n x^n                        & = 0 \\
    \sum\limits_{n = 2}^{ +\infty} n(n - 1) a_n (x^{n - 2} - x^n) - 4\sum\limits_{n = 2}^{ +\infty} na_n x^n - 4a_1x - 2\sum\limits_{n = 2}^{ +\infty} a_n x^n - 2a_0 - 2a_1x & = 0 \\
    \sum\limits_{n = 2}^{ +\infty} \left( n(n - 1) a_n (x^{n - 2} - x^n) - 4 na_n x^n - 2 a_n x^n\right) - 2a_0 - 6a_1x                                                       & = 0 \\
\end{align*}

Все коэффициенты при \(x_n\) равны \(0\):

\begin{align*}
     & \begin{cases}
        a_0 = a_2       \\
        6a_3 - 6a_1 = 0 \\
        (n + 1)(n + 2) a_{n + 2} - n(n - 1)a_{n} - 4na_n - 2a_n = 0
    \end{cases} \\
     & \begin{cases}
        a_0 = a_2 \\
        a_1 = a_3 \\
        (n^2 + 3n + 2) a_{n + 2} = (n^2 - n + 4n + 2)a_n
    \end{cases} \\
     & \begin{cases}
        a_0 = a_2 \\
        a_1 = a_3 \\
        (n^2 + 3n + 2) a_{n + 2} = (n^2 + 3n + 2)a_n
    \end{cases} \\
     & \begin{cases}
        a_0 = a_2 \\
        a_1 = a_3 \\
        a_{n + 2} = a_n
    \end{cases} \\
\end{align*}

Пусть \(a_0 = 1, a_1 = 0\). Тогда \(y = \sum\limits_{n = 0}^{ +\infty} x^{2n}\). При \(x \in ( - 1, 1)\) этот ряд сходится к \(\cfrac{1}{1 - x^2}\), иначе этот ряд не сходится.

Пусть \(a_0 = 0, a_1 = 1\). Тогда \(y = \sum\limits_{n = 0}^{ +\infty} x^{2n + 1}\). При \(x \in ( - 1, 1)\) этот ряд сходится к \(\cfrac{x}{1 - x^2}\), иначе этот ряд не сходится.

\section{}

Решить \(y'^4- y'^2=y^2\). Условия Коши \(y(-2 \sqrt 3)=-2 \sqrt 3\), рассмотрите \(x,y<0\) укажите \(y(-\pi/3)\).

\(y \equiv 0\) --- решение

\begin{align*}
    y'^4- y'^2                                                                       & = y^2 \\
    p : = y'                                                                                 \\
    p^4 - p^2                                                                        & = y^2 \\
    \pm \sqrt{p^4 - p^2}                                                             & = y   \\
    \pm \frac{(4p^3 - 2p)dp}{2\sqrt{p^4 - p^2}}                                      & = dy  \\
    \pm \frac{(4p^3 - 2p)dp}{2\sqrt{p^4 - p^2}}                                      & = pdx \\
    \pm \frac{(2p^3 - 1)dp}{\sqrt{p^4 - p^2}}                                        & = dx  \\
    \pm \frac{(2p^3 - 1)dp}{p\sqrt{p^2 - 1}}                                         & = dx  \\
    \pm \int \frac{(2p^3 - 1)dp}{p\sqrt{p^2 - 1}}                                    & = x   \\
    \pm \int \frac{2p^2dp}{\sqrt{p^2 - 1}} \mp \int \frac{dp}{p \sqrt{p^2 - 1}}      & = x   \\
    \pm \sqrt{p^2 - 1}p \pm \ln(2p + 2 \sqrt{p^2 - 1}) \pm \arcsin \frac{1}{|p|} + C & = x   \\
\end{align*}

\[
    \begin{cases}
        y = \pm \sqrt{p^2 - 1}p \\
        x = \pm \sqrt{p^2 - 1}p \pm \ln(2p + 2 \sqrt{p^2 - 1}) \pm \arcsin \frac{1}{|p|} + C
    \end{cases}
\]

Подставим условие Коши:

\begin{align*}
     & \begin{cases}
        -2 \sqrt{3} = \pm \sqrt{p^2 - 1}p \\
        -2 \sqrt{3} = -2 \sqrt{3} \pm \ln(2p + 2 \sqrt{p^2 - 1}) \pm \arcsin \frac{1}{|p|} + C
    \end{cases} \\
     & \begin{cases}
        -2 \sqrt{3} = \pm \sqrt{p^2 - 1}p \\
        0 = \pm \ln(2p + 2 \sqrt{p^2 - 1}) \pm \arcsin \frac{1}{|p|} + C
    \end{cases} \\
\end{align*}

Таким образом, есть два случая: \( p\in \{ \mp 2, \pm i\sqrt{3} \}\). Хочется верить, что в комплексные числа лезть не надо, поэтому рассмотрим \(p = \mp 2\).

\begin{align*}
    0                                         & = \pm \ln( \mp 4 + 2 \sqrt{3}) \pm \arcsin \frac{1}{2} + C \\
    0                                         & = - \ln(4 + 2 \sqrt{3}) - \arcsin \frac{1}{2}          + C \\
    \ln(4 + 2 \sqrt{3}) + \arcsin \frac{1}{2} & = C                                                        \\
\end{align*}

Найдём \(y( -\frac{\pi}{3})\):

\begin{align*}
     & \begin{cases}
        y = -\sqrt{p^2 - 1}p \\
        -\frac{\pi}{3} = -\sqrt{p^2 - 1}p - \ln(2p + 2 \sqrt{p^2 - 1}) - \arcsin \frac{1}{|p|} + \ln(4 + 2 \sqrt{3}) + \arcsin \frac{1}{2}
    \end{cases} \\
\end{align*}

\(p\approx 1.35475\). \(y \approx -1.2382\)

\section{}

$2y'^2(y-xy')=1$. Коши $y(1)=55/18$. Найти $y(1/9)$

\begin{align*}
    2y'^2(y-xy')       & = 1                              \\
    p : = y'                                              \\
    y                  & = px + C                         \\
    2p^2 (px + C - px) & = 1                              \\
    2p^2 C             & = 1                              \\
    C                  & = \frac{1}{2p^2}                 \\
    y                  & = px + \frac{1}{2p^2}            \\
    dy                 & = pdx + xdp - \frac{dp}{p^3}     \\
    pdx                & = pdx + xdp - \frac{dp}{p^3}     \\
    0                  & = xdp - \frac{dp}{p^3}           \\
    x                  & = \frac{1}{p^3}                  \\
    x^\frac{1}{3}      & = \frac{1}{p}                    \\
    y'                 & = \frac{1}{x^\frac{1}{3}}        \\
    \int y' dx         & = \int \frac{dx}{x^\frac{1}{3}}  \\
    y                  & = \frac{3}{2}x^{\frac{2}{3}} + C \\
\end{align*}

Подставим условие Коши:
\begin{align*}
    \frac{55}{18}      & = \frac{3}{2} + C \\
    \frac{55 - 27}{18} & = C               \\
    \frac{14}{9}       & = C               \\
\end{align*}

Найдём \(y(1 / 9)\):
\begin{align*}
    y(1 / 9) & = \frac{3}{2} \frac{1}{9^{2 / 3}} + \frac{14}{9} \\
             & = \frac{1}{2\cdot 3^{1 / 3}} + \frac{14}{9}      \\
\end{align*}

\end{document}