\documentclass[12pt, a4paper]{article}

% Math symbols
\usepackage{amsmath, amsthm, amsfonts, amssymb}
\usepackage{accents}
\usepackage{mathrsfs}
\usepackage{mathtools}

% Page layout
% \usepackage[left=1.5cm, right=1.5cm, top=2cm, bottom=2cm, bindingoffset=0cm, headheight=15pt]{geometry}
\usepackage{parskip, fancyhdr, a4wide}

% Font, encoding, russian support
\usepackage[russian]{babel}
\usepackage[sb]{libertine}
\usepackage{xltxtra}

% Miscellaneous
\usepackage{calc}
\usepackage{catchfilebetweentags}
\usepackage{enumitem}
\usepackage{etoolbox}
\usepackage{lastpage}
\usepackage{wrapfig}
\usepackage{xcolor}
\usepackage{bigints}

\providetoggle{useproofs}
\settoggle{useproofs}{false}

\pagestyle{fancy}
\lfoot{M3137y2019}
\rhead{стр. \thepage\ из \pageref{LastPage}}

\newcommand{\R}{\mathbb{R}}
\newcommand{\Q}{\mathbb{Q}}
\newcommand{\C}{\mathbb{C}}
\newcommand{\Z}{\mathbb{Z}}
\newcommand{\B}{\mathbb{B}}
\newcommand{\N}{\mathbb{N}}

\newcommand{\const}{\text{const}}

\newcommand{\teormin}{\textcolor{red}{!}\ }

\DeclareMathOperator*{\xor}{\oplus}
\DeclareMathOperator*{\equ}{\sim}
\DeclareMathOperator{\sign}{\text{sign}}
\DeclareMathOperator{\Sym}{\text{Sym}}
\DeclareMathOperator{\Asym}{\text{Asym}}

\DeclarePairedDelimiter{\ceil}{\lceil}{\rceil}
\DeclarePairedDelimiter{\abs}{\left\lvert}{\right\rvert}

\theoremstyle{plain}
\newtheorem{axiom}{Аксиома}
\newtheorem{lemma}{Лемма}[section]

\theoremstyle{remark}
\newtheorem*{remark}{Примечание}
\newtheorem*{exercise}{Упражнение}
\newtheorem*{corollary}{Следствие}
\newtheorem*{example}{Пример}
\newtheorem*{observation}{Наблюдение}

\theoremstyle{definition}
\newtheorem{theorem}{Теорема}[section]
\newtheorem*{definition}{Определение}
\newtheorem*{obozn}{Обозначение}

\newcommand{\dbltilde}[1]{\accentset{\approx}{#1}}
\newcommand{\intt}{\int\!}

% magical thing that fixes paragraphs
\makeatletter
\patchcmd{\CatchFBT@Fin@l}{\endlinechar\m@ne}{}
  {}{\typeout{Unsuccessful patch!}}
\makeatother

\newcommand{\get}[2]{
    \ExecuteMetaData[#1]{#2}
}

\newcommand{\getproof}[2]{
    \iftoggle{useproofs}{\ExecuteMetaData[#1]{#2proof}}{}
}

\newcommand{\getwithproof}[2]{
    \get{#1}{#2}
    \getproof{#1}{#2}
}

\newcommand{\import}[3]{
    \subsection{#1}
    \getwithproof{#2}{#3}
}

\newcommand{\given}[1]{
    Дано выше. (\ref{#1}, стр. \pageref{#1})
}

\renewcommand{\ker}{\text{Ker }}
\newcommand{\im}{\text{Im }}
\newcommand{\grad}{\text{grad}}
\newcommand{\defeq}{\stackrel{\text{def}}{=}}
\newcommand{\itemfix}{\leavevmode\makeatletter\makeatother}

\usepackage{sectsty}

\allsectionsfont{\raggedright}
\sectionfont{\normalfont\fontsize{14}{15}\selectfont}
\subsectionfont{\normalfont\fontsize{13}{15}\selectfont}
\usepackage{svg}
\svgpath{{images/}}
\allowdisplaybreaks

\makeatletter
\newenvironment{sqcases}{%
  \matrix@check\sqcases\env@sqcases
}{%
  \endarray\right.%
}
\def\env@sqcases{%
  \let\@ifnextchar\new@ifnextchar
  \left\lbrack
  \def\arraystretch{1.2}%
  \array{@{}l@{\quad}l@{}}%
}
\makeatother

\usepackage{tikz}
\usetikzlibrary{arrows.meta}
\usetikzlibrary{decorations.pathmorphing}
\usetikzlibrary{calc}
\usepackage{cases}

\lhead{Домашнее задание №7}
\lfoot{Михайлов Максим}
\cfoot{}
\rfoot{M3237}

\begin{document}

\section{$(x-y^2)y'=1$}

Найдём интегрирующий множитель $\mu$:

\begin{align*}
    \mu'_y P - \mu'_x Q & = (Q'_x - P'_y) \mu \\
\end{align*}

Пусть $\mu'_x \equiv0$

\begin{align*}
    \mu'_y P    & = (Q'_x - P'_y) \mu \\
    \mu'_y (-1) & = 1 \mu             \\
    \mu         & = e^{-y}
\end{align*}

Домножим исходное уравнение на $\mu$:

\begin{align*}
    -e^{-y} dx + e^{-y}(x-y^2) dy & = 0
\end{align*}

Найдем $u(x, y)$, такое что:
\[\begin{cases}
        u'_x = -e^{-y} \\
        u'_y = e^{-y}(x-y^2)
    \end{cases}\]

\begin{align*}
    u & = \int -e^{-y}dx + g(y) \\
      & = -e^{-y}x + g(y)       \\
\end{align*}

\begin{align*}
    u'_y                 & = e^{-y}(x-y^2)          \\
    (-e^{-y}x + g(y))'_y & = e^{-y}(x-y^2)          \\
    e^{-y}x + g(y)'_y    & = e^{-y}(x-y^2)          \\
    g(y)'_y              & = -e^{-y}y^2             \\
    g(y)                 & = \int -e^{-y}y^2dy + C  \\
    g(y)                 & = -e^{-y}(-y^2-2y-2) + C \\
\end{align*}

\begin{align*}
    u & = -e^{-y}x - e^{-y}(-y^2-2y-2) + C
\end{align*}

Ответ: $-e^{-y}x - e^{-y}(-y^2-2y-2) = C$

\section{$x-y/y'=2/y$}

\begin{align*}
    x - \frac{y}{y'}   & = \frac{2}{y}        \\
    \frac{y}{y'}       & = x - \frac{2}{y}    \\
    \frac{y}{y'}       & = \frac{xy - 2}{y}   \\
    \frac{y'}{y}       & = \frac{y}{xy - 2}   \\
    \frac{dy}{dx}      & = \frac{y^2}{xy - 2} \\
    (xy-2)dy           & = y^2dx              \\
    - y^2dx + (xy-2)dy & =  0                 \\
\end{align*}

Найдём интегрирующий множитель $\mu$:

\begin{align*}
    \mu'_y P - \mu'_x Q & = (Q'_x - P'_y) \mu \\
\end{align*}

Пусть $\mu'_x \equiv0$

\begin{align*}
    \mu'_y P       & = (Q'_x - P'_y) \mu \\
    \mu'_y (- y^2) & = (y + 2y) \mu      \\
    \mu'_y y       & = -3 \mu            \\
    \mu            & = \frac{1}{y^3}
\end{align*}

Домножим исходное уравнение на $\mu$:

\begin{align*}
    -\frac{1}{y}dx + \frac{xy-2}{y^3}dy & = 0 \\
\end{align*}

Найдем $u(x, y)$, такое что:
\[\begin{cases}
        u'_x =  -\frac{1}{y} \\
        u'_y = \frac{xy-2}{y^3}
    \end{cases}\]

\begin{align*}
    u = \int -\frac{1}{y}dx + g(y) \\
    u = -\frac{x}{y} + g(y)        \\
\end{align*}

\begin{align*}
    u'_y                                & = \frac{xy-2}{y^3}          \\
    \left(-\frac{x}{y} + g(y)\right)'_y & = \frac{xy-2}{y^3}          \\
    \frac{x}{y^2} + g'(y)               & = \frac{xy-2}{y^3}          \\
    g'(y)                               & = \frac{-2}{y^3}            \\
    g(y)                                & = \int \frac{-2}{y^3}dy + C \\
    g(y)                                & = \frac{1}{y^2} + C         \\
\end{align*}

\begin{align*}
    u = -\frac{x}{y} + \frac{1}{y^2} + C
\end{align*}

Ответ: $-\frac{x}{y} + \frac{1}{y^2} = C$

\section{$y'+y=xy^3$}

$y\equiv0$ --- решение.

Это уравнение Бернулли при $p(x) = -1, q(x) = x, \alpha = 3$. Выполним стандартную замену $t = y^{1-\alpha}$:

\begin{align*}
    t := y^{-2}                  & \quad t' = -\frac{2y'}{y^3} \\
    y'+y                         & = xy^3                      \\
    \frac{y'}{y^3}+\frac{1}{y^2} & = x                         \\
    -\frac{t'}{2} + t            & = x                         \\
    t'                           & = 2t - 2x                   \\
\end{align*}

Это линейное уравнение при $p(x) = 2, q(x) = -2x$

\begin{align*}
    t      & = \left(\int e^{-\int p} qdx\right) e^{\int p}            \\
           & = -2\left(\int e^{-\int 2} xdx\right) e^{\int 2}          \\
           & = -2\left(\int e^{-2x} xdx\right) e^{2x}                  \\
           & = -\left(-e^{-2x}x + \int e^{-2x}dx\right) e^{2x}         \\
           & = -\left(-e^{-2x}x - \frac{e^{-2x}}{2} + C \right) e^{2x} \\
           & = -\left(-x - \frac{1}{2} \right) + Ce^{2x}               \\
           & = x + \frac{1}{2} + Ce^{2x}                               \\
    y^{-2} & = x + \frac{1}{2} + Ce^{2x}
\end{align*}

Ответ: $y^{-2} = x + \frac{1}{2} + Ce^{2x}$ или $y\equiv0$

\section{$(xy^4-x)dx+(y+xy)dy=0$}

\begin{align*}
    (xy^4-x)dx+(y+xy)dy & = 0 \\
    x(y^4-1)dx+y(x+1)dy & = 0 \\
\end{align*}

Это уравнение с разделяющимися переменными. Хочется поделить на $(y^4-1)$ и $(x+1)$, но сначала надо рассмотреть случаи, когда они $\equiv0$.

\begin{align*}
    y^4 - 1                & \equiv 0 \\
    y                      & \equiv 1 \\
    y'                     & \equiv 1 \\
    (x^4 - x)dx + (y+xy)y' & = 0      \\
    0dx + 0                & = 0      \\
\end{align*}

Таким образом, решение $y \equiv 1$ подходит.

\begin{align*}
    x + 1               & \equiv 0  \\
    x                   & \equiv -1 \\
    x'                  & \equiv 0  \\
    x(y^4-1)x'+y(x+1)dy & = 0       \\
    0+y\cdot 0          & = 0       \\
\end{align*}

Таким образом, решение $x \equiv -1$ подходит.

\begin{align*}
    x(y^4-1)dx+y(x+1)dy               & = 0                       \\
    \frac{x}{x+1}dx+\frac{y}{y^4-1}dy & = 0                       \\
    \frac{x}{x+1}dx                   & = -\frac{y}{y^4-1}dy      \\
    \int \frac{x}{x+1}dx              & = -\int \frac{y}{y^4-1}dy \\
\end{align*}

\begin{align*}
    \int \frac{x}{x+1}dx & = \int dx - \int \frac{1}{x+1}dx \\
                         & = x - \ln|x+1|
\end{align*}

\begin{align*}
    t := \frac{y^2}{2}                                                                               \\
    \int \frac{y}{y^4-1}dy & = \int \frac{1}{4t^2 - 1} dy                                            \\
                           & = \int \frac{1}{(2t-1)(2t+1)} dy                                        \\
                           & = \frac{1}{2}\int \frac{1}{2t-1} dy - \frac{1}{2}\int \frac{1}{2t+1} dy \\
                           & = \frac{\ln|2t-1|}{4} -  \frac{\ln|2t+1|}{4}  + C                       \\
                           & = \frac{\ln|y^2-1|}{4} -  \frac{\ln|y^2+1|}{4} + C                      \\
\end{align*}

\begin{align*}
    x - \ln|x+1| & = - \frac{\ln|y^2-1|}{4} + \frac{\ln|y^2+1|}{4} + C
\end{align*}

Ответ: $y \equiv 1$ или $x \equiv -1$ или $x - \ln|x+1| = - \frac{\ln|y^2-1|}{4} + \frac{\ln|y^2+1|}{4} + C$

\section{$yy'+xyy''=x(y')^2+x^3$}

\begin{align*}
    yy'+xyy''                     & = x(y')^2+x^3     \\
    y(xy')'                       & = (xy')y' + x^3   \\
    y(xy')' - (xy')y'             & = x^3             \\
    \frac{y(xy')' - (xy')y'}{y^2} & = \frac{x^3}{y^2} \\
    \left(\frac{xy'}{y}\right)'   & = \frac{x^3}{y^2} \\
\end{align*}

:(

\section{$y''\cos y+(y')^2\sin y=y'$}

$y\equiv\const$ - подходит

\begin{align*}
    y''\cos y+(y')^2\sin y                            & = y'                         \\
    t := y'                                           & \quad y'' = t't              \\
    t't\cos y+t^2\sin y                               & = t                          \\
    t'\cos y+t\sin y                                  & = 1                          \\
    % t'                     & = \frac{1}{\cos y} - t\tg y \\
    \frac{t'}{\cos y}+\frac{t\tg y}{\cos y}           & = \frac{1}{\cos^2 y}         \\
    \frac{t'}{\cos y}+t\left(\frac{1}{\cos y}\right)' & = \frac{1}{\cos^2 y}         \\
    \left(\frac{t}{\cos y}\right)'                    & = \frac{1}{\cos^2 y}         \\
    \int \left(\frac{t}{\cos y}\right)' dy            & = \int \frac{1}{\cos^2 y} dy \\
    \frac{t}{\cos y} + C                              & = \tg y                      \\
    t                                                 & = \sin y + C\cos y           \\
    y'                                                & = \sin y + C\cos y           \\
    \frac{y'}{\sin y + C\cos y}                       & = 1                          \\
    \frac{dy}{\sin y + C\cos y}                       & = dx                         \\
    \int \frac{dy}{\sin y + C\cos y}                  & = \int dx                    \\
    \int \frac{dy}{\sin y + C\cos y}                  & = x + C_1                    \\
\end{align*}

\begin{align*}
    a := \tg\left(\frac{y}{2}\right) & \quad dy = \frac{2}{1 + a^2}da                                                                                                                                               \\
    \int \frac{dy}{\sin y + C\cos y}
                                     & = \int \frac{\frac{2}{1 + a^2}da}{\frac{2a}{1+a^2} + C\frac{1-a^2}{1+a^2}}                                                                                                   \\
                                     & = \int \frac{2da}{2a + C(1-a^2)}                                                                                                                                             \\
                                     & = \int \frac{2da}{2a + C - Ca^2}                                                                                                                                             \\
                                     & = 2\int \frac{C}{\left(-Ca+\sqrt{C^2+1}+1\right)\left(Ca+\sqrt{C^2+1}-1\right)} da                                                                                           \\
                                     & = 2\int \left(\frac{C}{2\sqrt{C^2+1}\left(Ca+\sqrt{C^2+1}-1\right)} - \frac{C}{2\sqrt{C^2+1}\left(Ca-\sqrt{C^2+1}-1\right)}\right) da                                        \\
                                     & = 2\int \left(\frac{C}{2\sqrt{C^2+1}\left(Ca+\sqrt{C^2+1}-1\right)} - \frac{C}{2\sqrt{C^2+1}\left(Ca-\sqrt{C^2+1}-1\right)}\right) da                                        \\
                                     & = \frac{\ln\left(\left|C\tg\left(\frac{y}{2}\right)+\sqrt{C^2+1}-1\right|\right)-\ln\left(\left|C\tg\left(\frac{y}{2}\right)-\sqrt{C^2+1}-1\right|\right)}{\sqrt{C^2+1}}+C_1 \\
    x                                & = \frac{\ln\left(\left|C\tg\left(\frac{y}{2}\right)+\sqrt{C^2+1}-1\right|\right)-\ln\left(\left|C\tg\left(\frac{y}{2}\right)-\sqrt{C^2+1}-1\right|\right)}{\sqrt{C^2+1}}+C_1 \\
\end{align*}

Ответ: $y\equiv C$ или $x = \frac{\ln\left(\left|C\tg\left(\frac{y}{2}\right)+\sqrt{C^2+1}-1\right|\right)-\ln\left(\left|C\tg\left(\frac{y}{2}\right)-\sqrt{C^2+1}-1\right|\right)}{\sqrt{C^2+1}}+C_1$

% Это линейное уравнение, $p(y) = -\tg y, q(y) = \frac{1}{\cos y}$

% \begin{align*}
%     t  & = \left(\int e^{-\int p} q\right) e^{\int p}                             \\
%        & = \left(\int \frac{e^{\int \tg ydy}}{\cos y} dy\right) e^{-\int \tg ydy} \\
%        & = \left(\int \frac{C_1\cos y}{\cos y}dy\right) \frac{C_2}{\cos y}        \\
%        & = C_1y \frac{C_2}{\cos y}                                                \\
%        & = C \frac{y}{\cos y}                                                     \\
%     y' & = C \frac{y}{\cos y}                                                     \\
%     x  & = \int C\frac{\cos y dy}{y}                                              \\
%      & = C\int \frac{\cos y dy}{y}                                              \\
% \end{align*}

\section{$y''''' - 6 y'''' + 9y'''=0$}

Cоставим характеристическое уравнение:

\begin{align*}
    \lambda^5 - 6\lambda^4 + 9\lambda^3 = 0 \\
\end{align*}

Корни - $0$ кратности $3$ и $3$ кратности $2$.

Ответ: $y = C_1 + C_2x + C_3x^2 + C_4e^{3x} + C_5xe^{3x}$

\end{document}